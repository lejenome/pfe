La croissance rapide des flux de données en quantité et en valeur a caussé le
besoin de trouver des solutions efficaces pour les exploiter et les anaylser.
Dans ce cadre, on trouve plusieurs sociétés se profilent de ces données comme
Facebook et Twitter dont le produit principale est le service d'analyse et
d'identification des patterns de comportement des utilisateurs. Ces services
dépend des technologies de de \textquote{Big Data} et \textquote{Data Mining}.

%Globalement, plusieurs sociétés dépendent de la collecte des grandes masses des
%données et de l'analyse des flux des données, notons Facebook, Twitter, \ldots.
%En effet, leur produit principale est le service d'analyse et d'identification
%des patterns de comportement des utilisateurs qui se substitue en le travail de
%\textquote{Big-Data} et \textquote{Data Mining}.

\begin{center}
\textbf{\textit{Je n’aime pas les chiffres! Je préfère les décortiquer pour
percevoir leur véritable signification.}} \linebreak
\hfill --- \textsc{Denis Dupouy}, \ \textit{Web Analyste}
\end{center}

Dans ce contexte, nous allons mettre en \oe{}uvre un système de
collection, d'analyse et de valorisation des données, c'est à dire de transformer
en des services significatifs et rentables.

Ce rapport présente le travail que nous avons effectué lors de notre stage au
sein de \textquote{Djagora Academy}. En effet, \textquote{Djagora Academy} est
un programme de mentorat qui accompagne ses participants vers l'exploration du monde
professionnel et entrepreneurial à l'aide d'une vaste communauté d'industriel,
professeur et porteur d'expérience.

\textquote{Djagora Academy} est basée sur l'idée d'avoir plusieurs équipes d'étudiants
qui travaillent collectivement sur des projets dont le but est la création des
startups.

Le projet \textquote{City Watch} est mené par une équipe de 8 membres que nous faisons partit
qui travaillent sur l'axe de créer un système de collecte et d'analyse
des données routières.

Le présent manuscrit est composé de cinq chapitres. Le premier chapitre met le
projet dans son contexte général. À ce niveau, nous présentons la méthode de
gestion du projet utilisée à laquelle nous avons fait recours pour le pilotage
de notre projet. Le reste du manuscrit présente l'élaboration des taches
accomplir pour atteindre les objectifs du projet.
