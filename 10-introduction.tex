\TODO{le context,problematique,Motivation}
\TODO{Démarche à suivre pour atteindre le but +outils}
\TODO{l'organisation du document}
\TODO{wath will we find}

Ce rapport présente le travail que nous avons effectué lors de notre stage au sein de
Djagora Academy

La capacité de collecte un montant massif des donnes est révolutionnaires.

Dans ce contexte,..

 Il y a plusieurs années, nous avons fait face à une révolution technologique
 dans tous les secteurs de l'industrie (Économie, banque, tourisme, transport,\ldots)


Cette révolution a créé un besoin de données.



Dans le cadre de notre projet de fin d’études nous allons mettre en œuvre un système de
détection,de collection, d'analyse des donnes,\ldots
 fin d’estimer le nombre de ces objets dans des lieux publics.
Le but de ce projet est de déterminer
le succès d’une exposition.

Le présent manuscrit est composé de cinq chapitres. Le premier chapitre met le
projet dans son contexte général. A ce niveau, nous présentons la société d’accueil,
ainsi que la méthode agile SCRUM à laquelle nous
avons fait recours pour la gestion de notre projet.
Le reste du manuscrit est scindé selon
le nombre des itérations définies selon la méthode SCRUM. Le deuxième chapitre
présente l’itération 0 qui définit le carnet du produit à développer. Le troisième et le
quatrième chapitre mettent l’accent sur le processus de développement itératif et
incrémentale défini dans la méthode SCRUM.