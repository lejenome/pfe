\TODO{context, problematique, Motivation}
\TODO{Démarche suivi pour atteindre le but + outils}
\TODO{organisation du doc}
\TODO{what will we find}

%La capacité de collecte un montant massif des donnes est révolutionnaires.

Il y a plusieurs années, nous avons fait face à une révolution technologique
dans tous les secteurs de l'industrie (Économie, banque, tourisme,
transport, \ldots). Cette révolution a créé un besoin de données et analyse.

\begin{center}
\textbf{\textit{Je n’aime pas les chiffres! Je préfère les décortiquer pour
percevoir leur véritable signification.}} \linebreak
\hfill --- \textsc{Denis Dupouy}, \ \textit{Web Analyste}
\end{center}

Dans ce contexte, nous allons mettre en \oe{}uvre un système de détection, de
collection et d'analyse des donnes, \ldots. À fin de les transformer en
information significative et rentable. Le but de ce projet est d'élaborer une
idée de startup qui se base sur le mécanisme créé.

Ce rapport présente le travail que nous avons effectué lors de notre stage au
sein de \textquote{Djagora Academy}. En effet \textquote{Djagora Academy} est
un organisme qui accompagne ses participants vers l'exploration du monde
professionnel et entrepreneurial à l'aide d'une vaste communauté d'industriel,
professeur et porteur d'expérience.


Le présent manuscrit est composé de cinq chapitres. Le premier chapitre met le
projet dans son contexte général. À ce niveau, nous présentons la société
d’accueil, ainsi que la méthode agile Scrum à laquelle nous avons fait recours
pour le pilotage de notre projet. Le reste du manuscrit est scindé selon le
nombre des itérations définies selon la méthode Scrum. Le deuxième chapitre
présente l'itération 0 qui définit le carnet du produit à développer. Le
troisième et le quatrième chapitre mettent l'accent sur le processus de
développement itératif et incrémentale défini dans la méthode Scrum.

\TODO{check report struct description}
