\TODO{le context,problematique,Motivation}
\TODO{Démarche à suivre pour atteindre le but +outils}
\TODO{l'organisation du document}
\TODO{wath will we find}



%La capacité de collecte un montant massif des donnes est révolutionnaires.



 Il y a plusieurs années, nous avons fait face à une révolution technologique
 dans tous les secteurs de l'industrie (Économie, banque, tourisme, transport,\ldots)
Cette révolution a créé un besoin de données et analyse.
 %citation


 « Je n’aime pas les chiffres ! Je préfère les décortiquer pour percevoir leur véritable signification. »

Denis Dupouy, Web Analyste







Dans ce contexte nous allons mettre en œuvre
un système de
détection,de collection et d'analyse des donnes,\ldots
 fin de les transformer en information significative et rentable.
Le but de ce projet est d'élaborer une idée de start-up qui se base sur le mécanisme créé

Ce rapport présente le travail que nous avons effectué lors de notre stage au sein de
Djagora Academy
En effet Djagora Academy est un organisme qui accompagne ces participant vers
l'exploration du monde professionnel et entreprenarial a l'aide d'une vaste communauté
d'industriel , professeur et porteur d'expérience.


Le présent manuscrit est composé de cinq chapitres. Le premier chapitre met le
projet dans son contexte général. A ce niveau, nous présentons la société d’accueil,
ainsi que la méthode agile SCRUM à laquelle nous
avons fait recours pour le pilotage de notre projet.
Le reste du manuscrit est scindé selon
le nombre des itérations définies selon la méthode SCRUM. Le deuxième chapitre
présente l’itération 0 qui définit le carnet du produit à développer. Le troisième et le
quatrième chapitre mettent l’accent sur le processus de développement itératif et
incrémentale défini dans la méthode SCRUM.

