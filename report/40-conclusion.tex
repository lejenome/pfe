\chapter*{Conclusion Générale}
\addcontentsline{toc}{chapter}{Conclusion Générale}

Centré sur le concept de travail en équipe, nous avons réalisé une plateforme nommée
\textquote{City Watch} qui offre plusieurs services en relation avec la collecte, le traitement, l’analyse
et visualisation des données routières. Basé sur la gestion de données, le développement de cette
plateforme repose sur deux volets à savoir: la collecte de données et le traitement et ana-
lyse de celles-ci. Pour y parvenir, nous avons fait recours à la méthode Scrum qui nous a
facilités largement l’organisation des tâches et de l’équipe de développement composé de
huit membres. Il s’agit d’une méthode itérative qui implique les clients (investisseurs
potentiels dans notre cas) dans les changements au niveau des listes des caractéristiques du
produit. Ainsi, pour chaque itération, une succession d’événements aura lieu ; de la fixation
des objectifs, passant par la planification des tâches, la production de la liste globale
de caractéristiques du produite jusqu'à l’évaluation et la rétrospective.

La contribution de
notre binôme par rapport à la réalisation de cette plateforme consiste à mettre en place
des services web d'échange et de stockage des données routières. Pour ce faire, nous avons fait
recours aux Langages PHP7, HTML5, JavaScript ECMAScript 2016, CSS3, et les technologies JSON, Lumen, POWER BI, etc. En conclusion,
s’impliquer au programme Djagora Academy nous a incités à renforcer nos compétences
techniques, managériales et de communication. Les formations que nous avons suivies et les
réunions avec des experts dans plusieurs domaines nous ont permis d’élargir nos réseaux
relationnels professionnels et ont augmenté par conséquent nos chances pour réussir notre
intégration dans le monde professionnel. De plus, le travail au sein d’une équipe nous a
appris à bien gérer et exploiter le temps, à être responsable, transparent, engagé et surtout
collaboratif.

Maintenant et plus que jamais, nous sommes bien motivées à l’accomplissement de
notre mission au cours du programme Djagora Academy pour aller jusqu'à l’implantation
de la start-up.
