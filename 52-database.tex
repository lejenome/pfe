\chapter{Schéma de la base de données}

Le schéma des taleaux de la base de données est présenté dans les taleaux
suivants. On va présenter seulement le schéma dans l'itération 1 et l'itération
3.

\section{Schéma de la base de données en Itération 1}
\label{sec:sprint1-database}

\begin{table}[H]
   \centering
   \begin{tabular}{|l|p{4cm}|p{8cm}|}
   \hline
   \textbf{Colonne} & \textbf{Type} & \textbf{Description} \\
   \hline
   \underline{id} & UNSIGNED INT &\\ \hline
   lat & DECIMAL(16, 14) &$latitude \in [-90, 90]$ \\ \hline
   lng & DECIMAL(17, 16) & $longitude \in [-180, 180]$\\ \hline
   last\_modified & TIMESTAMP & \\ \hline
   \end{tabular}
   \caption{Schéma du tableau ``positions'' --- Itération 1}
\end{table}

\section{Schéma de la base de données en Itération 2 \& 3}
\label{sec:sprint3-database}

\begin{table}[H]
   \centering
   \begin{tabular}{|l|p{4cm}|p{8cm}|}
   \hline
   \textbf{Colonne} & \textbf{Type} & \textbf{Description} \\
   \hline
   \underline{id} & UNSIGNED INT &\\ \hline
   \#device\_id & UNSIGNED INT &\\ \hline
   lat & DECIMAL(16, 14) &$latitude \in [-90, 90]$ \\ \hline
   lng & DECIMAL(17, 16) & $longitude \in [-180, 180]$\\ \hline
   created\_at & TIMESTAMP & \\ \hline
   \end{tabular}
   \caption{Schéma du tableau ``positions''}
\end{table}

\begin{table}[H]
   \centering
   \begin{tabular}{|l|p{4cm}|p{8cm}|}
   \hline
   \textbf{Colonne} & \textbf{Type} & \textbf{Description} \\
   \hline
   \underline{id} & UNSIGNED INT &\\ \hline
   name &VARCHAR(20) &\\ \hline
   type & VARCHAR(10) & smartphone, tablette, laptop, \ldots\\ \hline
   \#user\_id & UNSIGNED INT & \\ \hline
   \end{tabular}
   \caption{Schéma du tableau ``devices''}
\end{table}

\begin{table}[H]
   \centering
   \begin{tabular}{|l|p{4cm}|p{8cm}|}
   \hline
   \textbf{Colonne} & \textbf{Type} & \textbf{Description} \\
   \hline
   \underline{id} & UNSIGNED INT &\\ \hline
   username & VARCHAR(20) & \\ \hline
   password & VARCHAR(60) & \\ \hline
   email & VARCHAR(254) & \cite{email-length} \\ \hline
   created\_at& TIMESTAMP & \\ \hline
   updated\_at & TIMESTAMP & \\ \hline
   \end{tabular}
   \caption{Schéma du tableau ``users''}
\end{table}

\begin{table}[H]
   \centering
   \begin{tabular}{|l|p{4cm}|p{8cm}|}
   \hline
   \textbf{Colonne} & \textbf{Type} & \textbf{Description} \\
   \hline
   \underline{id} & UNSIGNED INT &\\ \hline
   \#device\_id & UNSIGNED INT & \\ \hline
   type & VARCHAR(50) & \\ \hline
   description & TEXT & \\ \hline
   lat & DECIMAL(16, 14) & $latitude \in [-90, 90]$\\ \hline
   lng & DECIMAL(17, 14) & $longitude \in [-180, 180]$\\ \hline
   photo\_id & CHAR(36) UNIQUE & UUID v4 de la photo enregistrée dans le dossier \verb|uploads| \\ \hline
   created\_at & TIMESTAMP & \\ \hline
   \end{tabular}
   \caption{Schéma du tableau ``reports''}
\end{table}

\begin{table}[H]
   \centering
   \begin{tabular}{|l|p{4cm}|p{8cm}|}
   \hline
   \textbf{Colonne} & \textbf{Type} & \textbf{Description} \\
   \hline
   \underline{id} & UNSIGNED INT &\\ \hline
   \#device\_id & UNSIGNED INT & \\ \hline
   lat & DECIMAL(16, 14) & $latitude \in [-90, 90]$\\ \hline
   lng & DECIMAL(17, 14) & $longitude \in [-180, 180]$\\ \hline
   intensity & UNSIGNED INT(1) & $intensity \in [0, 7]$, 8 niveaux d'intensité \\ \hline
   dir & UNSIGNED INT(3) & $direction \in [0\degree, 360\degree[$ \\ \hline
   created\_at & TIMESTAMP & \\ \hline
   \end{tabular}
   \caption{Schéma du tableau ``potholes''}
\end{table}

\begin{table}[H]
   \centering
   \begin{tabular}{|l|p{4cm}|p{8cm}|}
   \hline
   \textbf{Colonne} & \textbf{Type} & \textbf{Description} \\
   \hline
   \underline{id} & UNSIGNED INT &\\ \hline
   \#device\_id & UNSIGNED INT & \\ \hline
   lat & DECIMAL(16, 14) & $latitude \in [-90, 90]$\\ \hline
   lng & DECIMAL(17, 14) & $longitude \in [-180, 180]$\\ \hline
   operator & UNSIGNED INT(6) & \acrfull{MCC} + \acrfull{MNC} \\ \hline
   strength & UNSIGNED INT(3) & $strength \in [0db, 100db[$ \\ \hline
   type & UNSIGNED INT(2) & Représentation numeric du génération du réseau (GSM, UMTS, EDGE, LTE, \ldots) basé sur la réprésentation utilisée en Android~\cite{getNetworkType}. \\ \hline
   created\_at & TIMESTAMP & \\ \hline
   updated\_at & TIMESTAMP & \\ \hline
   \end{tabular}
   \caption{Schéma du tableau ``phonesignals''}
\end{table}

\begin{table}[H]
   \centering
   \begin{tabular}{|l|p{4cm}|p{8cm}|}
   \hline
   \textbf{Colonne} & \textbf{Type} & \textbf{Description} \\
   \hline
   \underline{id} & UNSIGNED INT &\\ \hline
   \#device\_id & UNSIGNED INT & \\ \hline
   lat & DECIMAL(16, 14) & $latitude \in [-90, 90]$\\ \hline
   lng & DECIMAL(17, 14) & $longitude \in [-180, 180]$\\ \hline
   dir & UNSIGNED INT(3) & $direction \in [0\degree, 360\degree[$ \\ \hline
   created\_at & TIMESTAMP & \\ \hline
   \end{tabular}
   \caption{Schéma du tableau ``speedbumps''}
\end{table}
\begin{table}[H]
   \centering
   \begin{tabular}{|l|p{4cm}|p{8cm}|}
   \hline
   \textbf{Colonne} & \textbf{Type} & \textbf{Description} \\
   \hline
   \underline{id} & UNSIGNED INT &\\ \hline
   email & VARCHAR(254) & \cite{email-length} \\ \hline
   phone & VARCHAR(20) & NULL \\ \hline
   message & VARCHAR(255) & \\ \hline
   archived & TINY INT(1) & DEFAULT 0 \\ \hline
   created\_at & TIMESTAMP & \\ \hline
   updated\_at & TIMESTAMP & \\ \hline
   \end{tabular}
   \caption{Schéma du tableau ``feedbacks''}
\end{table}
