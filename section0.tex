\section{Sprint 0}
\subsection{Présentation du contexte}

\subsection{Présentation de la société}
Djagora Academy est un programme dont l’objectif est la mise en pratique des 
aspects  philoso-phiques liés au Leadership et à l’Entrepreneurship.

Il est à noter que le Leadership ou/et l’entrepreneurship inefficace/s demeure/ent un obstacle 
majeur aux actions industrielles, économiques, humanitaires, etc., efficaces. 
Basé sur la méthodologie dite «le Leadership Diamond®», développée par
Dr. Peter Koestenbaum et présentée à la Faculté des Sciences de Sfax Dr. Mahamouda
Salouhou (visiter think.factorycampus.net pour plus de détails), L’académie 
Djagora est vue comme un programme de mentorat visant accompagner des étudiants 
(mentorés) en année terminale par des universitaires et des industriels (mentors) 
avec le soutien de plusieurs compagnies internationales, telles que l’European 
Center for Leadership and Entrepreneurship Education (Eclee France, Eclee USA), 
Djagora University (Sénégal), NorthStar Paradigm Education (USA), Continental 
(Allemagne), Sivantos (Allemagne), Yousoft-IT (Tunisie), Factory Campus (Tunisie)
dans la réalisation de projets d’intérêt publique.

Considérés comme des idées de startups, les projets proposés par
le comité de pilotage de l'Académie Djagora se substituent aux projets de fin
d’études.
La durée de réalisation d’un projet s’étale sur cinq mois. Cette période
correspond à un cycle d’accélération de startups. Tout au long de ce cycle,
les mentorés suivront un programme de formations généralement certifiantes.

Mis-à-part le programme de mentorat et le cycle d’accélération de startups, 
l’Académie Djagora déclare le défi et organise une compétition tout au long
du cycle d’accélération de startups. Nommée «Bourse des startups», cette 
compétition unique en son genre, dont l’idée est attribuée à Factory Campus,
est une bonne opportunité pour tous les intervenants de l’Académie Djagora.
La «Bourse des startups» vise joindre l’utile du modèle financier et des retombées 
économiques de la bourse et de l’entrepreneuriat en général à l’agréable des 
bonnes pratiques et cultures du défi, de la concurrence, du challenge et du 
leadership. 
\subsection{Présentation du sujet}
\subsection{Présentation des outils utilisés}
\subsection{Backlog générale}
\subsection{Modélisation UML}
