\section{Sprint 0}
\subsection{Présentation du contexte}

\subsection{Présentation de la société}
Djagora Academy est un programme dont l’objectif est la mise en pratique des 
aspects  philoso-phiques liés au Leadership et à l’Entrepreneurship.

Il est à noter que le Leadership ou/et l’entrepreneurship inefficace/s demeure/ent un obstacle 
majeur aux actions industrielles, économiques, humanitaires, etc., efficaces. 
Basé sur la méthodologie dite «le Leadership Diamond®», développée par
Dr. Peter Koestenbaum et présentée à la Faculté des Sciences de Sfax Dr. Mahamouda
Salouhou (visiter think.factorycampus.net pour plus de détails), L’académie 
Djagora est vue comme un programme de mentorat visant accompagner des étudiants 
(mentorés) en année terminale par des universitaires et des industriels (mentors) 
avec le soutien de plusieurs compagnies internationales, telles que l’European 
Center for Leadership and Entrepreneurship Education (Eclee France, Eclee USA), 
Djagora University (Sénégal), NorthStar Paradigm Education (USA), Continental 
(Allemagne), Sivantos (Allemagne), Yousoft-IT (Tunisie), Factory Campus (Tunisie)
dans la réalisation de projets d’intérêt publique.

Considérés comme des idées de startups, les projets proposés par
le comité de pilotage de l'Académie Djagora se substituent aux projets de fin
d’études.
La durée de réalisation d’un projet s’étale sur cinq mois. Cette période
correspond à un cycle d’accélération de startups. Tout au long de ce cycle,
les mentorés suivront un programme de formations généralement certifiantes.

Mis-à-part le programme de mentorat et le cycle d’accélération de startups, 
l’Académie Djagora déclare le défi et organise une compétition tout au long
du cycle d’accélération de startups. Nommée «Bourse des startups», cette 
compétition unique en son genre, dont l’idée est attribuée à Factory Campus,
est une bonne opportunité pour tous les intervenants de l’Académie Djagora.
La «Bourse des startups» vise joindre l’utile du modèle financier et des retombées 
économiques de la bourse et de l’entrepreneuriat en général à l’agréable des 
bonnes pratiques et cultures du défi, de la concurrence, du challenge et du 
leadership. 
\subsection{Présentation du sujet}
En prenant départ d'un récepteur de position, le sujet mit à résoudre les 
problématiques localisations de multiple individu,la réalisation 
d'un module d'itinéraire pour les véhicules vers enfin la création
d'un mécanisme qui donne la main à effectuer divers types de rapport
dans une carte.
\subsection{Présentation des outils utilisés}
\subsubsection{SVN}
Un logiciel de gestion de versions est un logiciel de gestion
de configuration permettant
de stocker des informations pour une ou plusieurs ressources
informatiques permettant de
récupérer toutes les versions intermédiaires des ressources,
ainsi que les différences entre
les versions.


Subversion (en abrégé SVN) est un logiciel de gestion de versions, distribué sous licence
Apache et BSD.
VisualSVN est un plug-in d’intégration de Subversion de qualité professionnelle. Les
principaux avantages de VisualSVN sont les suivants :

* Fiabilité imbattable : Visual Studio ne s’arrêtera jamais ni ne s’arrêtera à cause
de VisualSVN.
* Intégration transparente : VisualSVN gère automatiquement les fichiers ajoutés
ou renommés et reflète ces opérations sur Subversion.
* Statut en temps réel : VisualSVN suit attentivement et affiche toutes les modifi-
cations apportées à votre copie de travail.
* Courbe d’apprentissage courte : VisualSVN utilise les boîtes de dialogue Tortoi-
seSVN et fournit un assistant intelligent pour mettre vos sources sous Subversion.
VisualSVN Server vous permet d’installer et de gérer facilement un serveur Subver-
sion entièrement fonctionnel sur la plate-forme Windows. est utile tant pour les petites
entreprises que pour les entreprises.
\subsubsection{Android}
Android est un système d'exploitation conçu en 2007 par la societé android,
start-up racheté par Google . Android est ``Open Source`` ca veut dire on peut
lire le code de ce logiciel le modifier et le redistribuer.

bases que nous utilisons:
\paragraph{IDE:}
C'est un logiciel dont son objectif est de faciliter le développement 
Il est toujours possible de développer une application sans IDE . 
l'IDE est constitué d'outils dont au moins un éditeur de texte. 
on utilisera dans notre projet l'IDE Android Studio, l'IDE privilège 
de Google . Android Studio comporte l'auto-complétion la génération
automatique  du code des outils de compilation et débogage et plusieurs
autres services qui permettent de développer une application facilement et
rapidement.
\paragraph{SDK:}
C’est une abréviation qui peut faire référence a :
Software Developpement Kit. les application android sont 
développé en JAVA un appareil sous android  ne comprend pas
le langage JAVA il comprend une variante de JAVA adapté pour lui.
on a recours ici SDK c'est ensemble un outil permettant de développer 
dans une  cible particulière une SDK android est alors un ensemble d'outil
qui permettent de développer des application pour android.
\paragraph{Activity:}
Un utilisateur habile d'android remarque que lors de l'exploitation d'une
application android qu'il est en train de naviguer entre des fenêtres et 
l'application ne afficher qu'une seule fenêtre a la fois  ces fenêtre la sont 
des activités on peut différencier ces activité a travers leur interface 
graphique ceci s'applique sur la plupart des application android car il y a
des applications qui contiennent pas d'activités. un première idée qui nous
frappe la tète c'est que une activité est un conteneur d'élément graphique qui 
constitue un interface graphique. Alors que ne non une activité n'est pas 
seulement une interface graphique mais elle va établir les liens entre l'interface 
graphique et la logique programma tique de plus l'activité contient des 
informations sur le statut actuel de l'application qui s'appelle le contexte
ce contexte permet de faire la liaison entre le système android et les autres 
activities de l'application.
\subparagraph{États d'activities:}
le système android met en place un système priorités entre application 
par exemple l'utilisateur est en train de naviguer sur internet et écouter 
de la musique il reçois un appel comme l'application qui gère les appel est 
une application plus prioritaire elle prend la du navigateur et le lecteur 
musique pour que l'utilisateur puisse répondre a son appel. Si une application
consomme trop de ressources et peut bloquer le fonctionnement du système android,
Android arrêtera cette application. et aussi comme expliqué précédemment 
les activités sont gères a partir d'un système de pile d'activités .
D'où l'apparition de plus d'un état qui sont centré sur l'activité.

On peut différencier ces états par leur visibilité :
\begin{itemize}
 \item État Active
 \item État Paused 
 \item État Stopped
\end{itemize}

\subparagraph{Cycle de vie d'une activité :}

une activité n'a pas de contrôle sur son état . 
son état change suivant un cycle rythmique entre l
e système android et les autres application (un système quasi 
dépendant sur des priorités comme expliqué précédemment) la figure qui suit 
explique le cycle de vie d'une activité. (Les états sont représenter comme des 
méthodes parce que lors de la programmation ces états sont interroges par le nom 
de ces méthodes.

% Diagram of Android activity life cycle
% Author: Pavel Seda 
% Drawing part, node distance is 1.5 cm and every node
% is prefilled with white background
\begin{figure}[H]
 \centering
 \footnotesize

\begin{tikzpicture}[node distance=1.5cm,
    every node/.style={fill=white, font=\sffamily}, align=center]
  % Specification of nodes (position, etc.)
  \node (start)             [activityStarts]              {L'activité démarre};
  \node (onCreateBlock)     [process, below of=start]          {onCreate()};
  \node (onStartBlock)      [process, below of=onCreateBlock]   {onStart()};
  \node (onResumeBlock)     [process, below of=onStartBlock]   {onResume()};
  \node (activityRuns)      [activityRuns, below of=onResumeBlock]
                                                      {Activity is running};
  \node (onPauseBlock)      [process, below of=activityRuns, yshift=-1cm]
                                                                {onPause()};
  \node (onStopBlock)       [process, below of=onPauseBlock, yshift=-1cm]
                                                                 {onStop()};
  \node (onDestroyBlock)    [process, below of=onStopBlock, yshift=-1cm] 
                                                              {onDestroy()};
  \node (onRestartBlock)    [process, right of=onStartBlock, xshift=4cm]
                                                              {onRestart()};
  \node (ActivityEnds)      [startstop, left of=activityRuns, xshift=-4cm]
                                                        {Le processus est tué};
  \node (ActivityDestroyed) [startstop, below of=onDestroyBlock]
                                                    {l'activité est arrêtée};     
  % Specification of lines between nodes specified above
  % with aditional nodes for description 
  \draw[->]             (start) -- (onCreateBlock);
  \draw[->]     (onCreateBlock) -- (onStartBlock);
  \draw[->]      (onStartBlock) -- (onResumeBlock);
  \draw[->]     (onResumeBlock) -- (activityRuns);
  \draw[->]      (activityRuns) -- node[text width=4cm]
                                   {Une autre activité s'intercole devent notre activité} (onPauseBlock);
  \draw[->]      (onPauseBlock) -- node {Notre activité n'est plus visible}
                                   (onStopBlock);
  \draw[->]       (onStopBlock) -- node {L'activité est arrêtée par le système ou l'utilisateur} (onDestroyBlock);
  \draw[->]    (onRestartBlock) -- (onStartBlock);
  \draw[->]       (onStopBlock) -| node[yshift=1.25cm, text width=3cm]
                                   {L'activité revientsur le devant de la scène}
                                   (onRestartBlock);
  \draw[->]    (onDestroyBlock) -- (ActivityDestroyed);
  \draw[->]      (onPauseBlock) -| node(priorityXMemory)
                                   {Priorité élevée $\rightarrow$ plus mémoire}
                                   (ActivityEnds);
  \draw           (onStopBlock) -| (priorityXMemory);
  \draw[->]     (ActivityEnds)  |- node [yshift=-2cm, text width=3.1cm]
                                    {L'utilisateur retourne vers l'activité}
                                    (onCreateBlock);
  \draw[->] (onPauseBlock.east) -- ++(2.6,0) -- ++(0,2) -- ++(0,2) --
     node[xshift=1.2cm,yshift=-1.5cm, text width=2.5cm]
     {L'activité revient sur le devant de la scéne}(onResumeBlock.east);

  \end{tikzpicture}
  \caption{Diagramme de cycle de vie d'activite Android}
  \captionsource{Pavel Seda, \TeX example.net [Modifié]}{http://www.texample.net/tikz/examples/android/}
  \label{fig:android-activity}
\end{figure}

\subparagraph{Comment choisir l'SDK optimale :}
Une SDK permet l'application de marcher sur la version android visé
et les versions ultérieure il  a noté de prendre en considération le taux des 
utilisateurs visé par cette application. aussi il faut travailler avec une SDK 
digne de confidence qui n'a pas de problème ou bug qui peuvent  bloquer ou arrêter
le fonctionnement de l'application l'SDK choisi doit pouvoir supporter les 
fonctionnalité offerte par l'application si on va utiliser une fonctionnalité
qui utilise les empreinte l'SDK dont on a travailler l'application doit supporter
cette fonctionnalité lorsque le travail sur une application est en groupe il est 
mieux que tous ce groupe utilise la même SDK pour éviter tous problème de 
compatibilité et conflit entre versions de SDK  donc il faut choisir une SDK 
qui est populaire en utilisation et qui est stable . Dans notre projet on va 
utiliser SDK 23 qui vise la version android 6.0 ayant un taux d'utilisateur 
qui est 4.79 % 
des utilisateur d'android notons que cet SDK comporte
la fonctionnalité d'android les plus récentes et qui est stable.
\subsubsection{Web Services}

\subsubsection{Lumen}

\subsection{Backlog générale}

\subsection{Modélisation UML}
