\section{Itération 0: ( 2/20/2017 - 2/21/2017 )}

\subsection{Objectif de l'itération}

Avant de commencer la première itération du Scrum, une période de temps à été
consacrée pour préparer ce qui est nécessaire au lancement du projet dans de
bonnes conditions. Cette période est souvent nommé l'itération 0 du projet.
Elle est consacrée généralement à la recherche bibliographique, aux choix
technologiques et à la mise en place de l'environnement de développement. Autre
que ces préparatifs, c'est dans cette itération que nous définissons le Backlog
de la plateforme, ainsi que le nombre d'itérations nécessaires et la durée de
l'itération. Il s'agit aussi d'une période de formation pour les membres de
l'équipe sur tous les environnements et les technologie à utiliser au cours du
montage du produit.

\begin{figure}[htbp]
  \centering
  \includegraphics[width=0.8\textwidth]{citywatch-architecture}
  \caption[Flux des information Géographiques en CityWatch]
  {Flux des informations Géographiques dans le système de localisation}
  \label{fig:citywatch-architecture}
\end{figure}

\subsection{Backlog générale}

La première étape de la méthode Scrum consiste à préparer un carnet du produit
(Product Backlog) qui présente la liste des tâches à effectuer durant le
développement du projet qui sera répartie en des itérations. Le rôle du
Product Owner est important dans cette phase de développement parce qu'il devra
faire l'exercice de prioriser ses demandes selon des critères respectant la
mission et les objectifs de son produit. En précisant la valeur de priorité, il
estime l'impact et le retour sur investissement qu'aura chacun des items dans
le carnet du produit. Il y a donc effectivement eu un gros travail d'échanges
et discussions avec le client pour comprendre tout le cahier des charge
initial, C'est comme ça que le Backlog a été défini.

Les valeurs d'affaires sont définies comme suit :
\begin{description}[align=right,labelwidth=1cm]
    \item [1] Définie une haute priorité, affectée pour les spécifications
        importantes, exigées pour le produit.
    \item [2] Définie une priorité importante mais pas exigée.
    \item [3] Définie une priorité moyenne ou parfois faible.
\end{description}

La figure~\ref{fig:product-backlog} présente une version simplifiée du Product
Backlog complète des 3 premières itérations. Nous représentons notre
participation par les cercles du texte souligné.

\usetikzlibrary{mindmap,shadows}
\newcommand*{\info}[4][16.3]{%
  \node [ annotation, #3, scale=0.65, text width = #1em,
          inner sep = 2mm ] at (#2) {%
  \list{$\bullet$}{\topsep=0pt\itemsep=0pt\parsep=0pt
    \parskip=0pt\labelwidth=8pt\leftmargin=8pt
    \itemindent=0pt\labelsep=2pt}%
    #4
  \endlist
  };
}
\begin{figure}[htbp]
%\usepackage{dtklogos}
\hspace{-9ex}
\begin{tikzpicture}[every annotation/.style={draw, fill=white, font=\Large}]
    \renewcommand{\href}[2]{#2}

    \path[mindmap,
          concept color=black!40,
          text=white,
          every node/.style={
              concept,
              circular drop shadow,
              execute at begin node=\hskip0pt,
          },
          %grow cyclic,
          root/.style={
              concept color=black!40,
              fill=white, line width=1ex, text=black,
              font=\footnotesize\bfseries,
              text width=7em},
          level 1 concept/.append style={
              font=\normalsize\bfseries,
              sibling angle=50,
              text width=7.7em,
              level distance=12.5em,
              inner sep=0pt},
          level 2 concept/.append style={
              font=\footnotesize\bfseries,
              level distance=8em},
          ours/.append style = {
              line width=0.5ex,
              concept color = black,
          }
    ]

    node[root] {Platforme CityWatch} [clockwise from=0]

    child[concept color=blue!70] {
        node {\ul{Gestions des Rapports}} [clockwise from=50]
        child { node { \ul{Consultation} } }
        child { node { \ul{Declaration} } }
    }
    child[concept color=green!40!black] {
        node[concept] {\ul{Information Routiere}} [clockwise from=30]
        child { node[concept] {\ul{Secousses}} }
        child { node[concept] {\ul{Ralentisseurs}}}
        child { node[concept] {Emboutiage}}
    }
    child[concept color=blue] {
        node[concept] {\ul{Itineraire}} [clockwise from=305]
        child { node[concept] {\ul{Localisation instantanee}} }
        child { node[concept] {\ul{Projection du Trajectoire}} }
        child { node[concept] {\ul{Historie des Trajectoires}} }
    }
    child[concept color=red!60!black] {
        node[concept] {Assitance} [clockwise from=235]
        child { node[concept] {Alert des secouces} }
        child { node[concept] {Alert des ralentisseurs} }
    }
    child[concept color=orange] {
        node[concept] {\ul{Tableau de Bord}} [counterclockwise from=100]
        child { node[concept] {\ul{Carte en Temps Reel}}}
        child { node[concept] {\ul{uthentification}} }
        child { node[concept] {\ul{Filtrages}} }
    }
    child[concept color=yellow!60!black] {
        node[concept] (Blogs) {\ul{Analyse et Visualization}} [clockwise from=135]
        child { node[concept] {Visualisation des données}}
        child { node[concept] {\ul{Tableau du board interactive}}}
    }
    child[concept color=red] {
        node[concept] {Information Réseau} [clockwise from=110]
        child { node[concept] {Signal Réseau} }
        child { node[concept] {Génération Réseau} }
        child { node[concept] {Opérateur Réseau} }
    };
    %\info{goForum.north east}{above,anchor=west,xshift=1em}{%
    %  \item[] Seit 2008
    %  \item 68\,444 Beiträge
    %  \item 13\,715 Themen
    %  \item 5\,532 registrierte Nutzer
    %}
    %\info{LaTeXForum.north west}{above,anchor=south}{%
    %  \item[] Seit 2008
    %  \item 81\,991 Beiträge
    %  \item 21\,026 Themen
    %  \item 13\,354 registrierte Nutzer
    %}
    %\info[8]{LaTeXArtikel.west}{below,anchor=north east,xshift=3em,yshift=-2em}{%
    %  \item 115 Artikel
    %}
    %\info[11]{LaTeXNews.south west}{below,anchor=north}{%
    %  \item 240 Meldungen
    %}
    %\info[9]{TikZGalerie.south}{below,anchor=north}{%
    %  \item[] Seit 2006
    %  \item 172 Autoren
    %  \item 384 Beispiele
    %}
    %\info[15]{goWiki.south}{below,anchor=north,xshift=3em}{%
    %  \item 152 erklärte Konzepte, Befehle und Pakete
    %}
    %\info{TeXweltQA.south east}{above,anchor=north west}{%
    %  \item[] Seit 2013
    %  \item 1\,710 Fragen
    %  \item 2\,151 Antworten
    %  \item 479 registrierte Nutzer
    %}
    %\info[8]{TeXweltBlog.south}{below,anchor=north,xshift=2em}{%
    %  \item[] Seit 2013
    %  \item 14 Autoren
    %}
    %\info[9]{PGFPlots.south west}{anchor=north east,xshift=1em}{%
    %  \item 14 Autoren
    %  \item 59 Beispiele
    %}
    %\info[6]{Planet.west}{anchor=east}{%
    %  \item 46 Blogs
    %}
    %\info[14]{TeXnique.east}{anchor=west,xshift = 0.5em}{%
    %  \item[] 2015, aufgrund Idee mit französischen
    %          \TeX-Freunden nach der TUG Damstadt, experimentell
    %}
    %\info[16]{Cookbook.east}{anchor=south west}{%
    %  \item[] Ab 10/2015, soll ca. 100 Beispiele aus
    %          dem \LaTeX\ Cookbook zeigen, sowie
    %          Community-Rezepte
    %}
\end{tikzpicture}
\caption{Objective du produit}
\label{fig:product-backlog}
\end{figure}


Le tableau~\ref{tab:product-backlog} représente la liste plus détaillé des cas
d'utilisations que nous allons implémentés pendant les 3 premières itérations.
Ils suivent la forme:
\begin{displayquote}
    En tant que \textbf{X}, je peux \textbf{Y} Afin d'\textbf{Z}.
\end{displayquote}
ou aussi la forme:
\begin{displayquote}
    En tant que \textbf{X}, je veux qu'il soit possible de \textbf{Y} pour \textbf{Z}.
\end{displayquote}

Dans notre liste des cas d'utilisation, l'acteur initiateur (\textbf{X}) est
toujours l'utilisateur.

\Needspace{5\baselineskip}
\begin{center}
    \footnotesize
    \setlength\LTleft{-20pt}
    \begin{longtable}{| l | p{3.5cm} | p{5.5cm} | p{5cm} | l |}
 \caption{Backlog du Produit}
 \label{tab:product-backlog} \\

 \hline
 \textbf{ID} & \textbf{Cas d'utilisation} & \textbf{Je veux qu'il soit possible de} & \textbf{Pour} & \textbf{Priorité} \\ \hline
 \endhead

 \hline \multicolumn{5}{|r|}{{Continué en page suivante$\dotsc$}} \\ \hline
 \endfoot

 \hline \hline
 \endlastfoot

\hline
%1 & Gestion du Trajectoire & démarrer le localisation & Avoir un feedback dans l'application et sur le site web & 1 \\ \cline{3-5}
%  &                        & arrêter le localisation & Avoir un feedback dans l'application & 1 \\ \hline
%2 & Gestion de Rapports    & choisir entre une variété de problèmes à déclarer depuis l'application & Avoir un feedback sur le site web & 1 \\ \cline{3-5}
%  &                        & choisir l'emplacement du problème à déclarer dans une carte & Avoir un feedback sur le site web & 1 \\ \cline{3-5}
%  &                        & ajouter une description ou une image au problème & Avoir un feedback sur le site web & 2 \\ \hline
%3 & Consulter la carte     & consulter la carte depuis l'application & Voir la carte dans l'application & 2 \\ \hline
%4 & Compte                 & consulter le site web sans avoir un compte & Consulter le site web avec un minimum d'informations & 1 \\ \cline{3-5}
%  &                        & créer un compte & Avoir un compte personnel & 2 \\ \hline
%5 & Groupement des rapports& voir les rapports en groupe lors d'un zoom out & Avoir une vision globale sur le nombre des rapports & 1 \\ \hline
%6 & Déclarer un rapport    & déclarer un rapport à partir du site web & Avoir accès à une page rapport comme celle de l'application & 1 \\ \hline
% \hline \ldots & \ldots & \ldots & \ldots & \ldots \\ \hline

%
1 & Gestion du Trajectoire & Activer le localisation & Enregistrer le trajectoire & 1 \\ \cline{3-5}
&                          & Consulter le dashboard  & Afficher le trajectoire instantané & 1 \\ \cline{3-5}
&                          & Filtrer le trajectoire  & Consulter l'histoire des trajectoires & 2 \\ \hline
2 & Gestion des Rapports   & Déclarer un rapport depuis le site & Avoir un rapport dans la carte & 1 \\ \cline{3-5}
&                          & Choisir un emplacement & Afficher et filtrer par catégorie & 1 \\ \cline{3-5}
&                          & Ajouter une image et commentaire & Enrichir le rapport & 1 \\ \cline{3-5}
&                          & Accéder à formulaire depuis l'application mobile & rapporter instantané et facilement & 2 \\ \cline{3-5}
&                          & Activer/Désactiver le groupement des marqueurs & Avoir une vision sur le nombre des rapports plus claire & 3 \\ \hline
3 & Tableau de bord (Dashboard) & Consulter le dashboard & Visualiser diffèrent type de données comme marqueurs et zones & 1 \\ \cline{3-5}
&                               & Manipuler la légende    & Filtrer les données affichés & 1 \\ \hline
4 & Comptes                     & Créer un compte & Pouvoir enregistrer des données personnels (trajectoire) & 3 \\ \cline{3-5}
&                               & Se connecter & Accéder au données routières privées (trajectoire) & 3 \\ \cline{3-5}
&                               & Visiter le dashboard sans compte & Consulter une version minimale des fonctionnalités & 3 \\
\hline
\end{longtable}
\end{center}

\subsection{Préparation de l'environnement du travail}

\subsubsection{Environnement logiciel}

La réalisation de ce projet nécessite l'ensemble de ces logiciels et
bibliothèques:

\paragraph{Android studio}
est un environnement de développement pour développer des applications Android.
Il est basé sur IntelliJ IDEA.\\ Android studio permet principalement d'éditer
les fichiers Java et les fichiers de configuration d'une application Android.Il
propose entre autres des outils pour gérer le développement d'applications
multilingues et permet de visualiser la mise en page des écrans sur des écrans
de résolutions variées simultanément

\paragraph{PHPStorm}
il est un environnement de développement intégré (IDE) dédié pour la
programmation en PHP. Il regroupe sous une interface conviviale un éditeur de
texte, un interpréteur et un débogueur\ldots. Ceci permet de faciliter la tâche de
programmation et de maximiser la productivité des développeurs.

\paragraph{Vim}
Vim est un éditeur de texte pour le terminal sous GNU/Linux. Il a une
stabilité exemplaire et ne cesse d'être amélioré. Malgré son austérité, c'est
un outil très puissant qui n'a rien à envier aux éditeurs graphiques comme
Gedit, Kate ou Mousepad. Il est très apprécié des développeurs pour toutes ses
fonctions qui en font un très bon IDE (coloration syntaxique de 200 langages,
complétion automatique, comparaison de fichiers, recherche évoluée, …) et est
extensible par des scripts.
\paragraph{Composer}

Le logiciel Composer est un gestionnaire de dépendances sous licence libre
écrit en PHP\@. Il permet à ses utilisateurs de déclarer et d'installer les
bibliothèques dont le projet principal a besoin.

\subsubsection{environnements matériels}

Pour le développement de ce projet il était demandé d'avoir :
\begin{itemize}
 \item Serveur Local pour les tests
     \begin{itemize}
         \item Windows 10
         \item Wamp 3
         \item Apache 2.4
         \item PHP 7.0
         \item MySQL 5.7
     \end{itemize}
 \item Serveur de production:
     \begin{itemize}
      \item Linux
      \item Accès FTP
      \item Apache 2.4
      \item PHP 7.0
      \item MySQL 5.5
     \end{itemize}
\end{itemize}

\subsection{Conclusion}

L'itération 0 à permis de préparer le terrain pour une bonne entame de
développement. Mis à part la mise en place des environnement de travail et les
formations que nous avons suivi sur les technologies, cette itération a permis
d'élaborer le carnet de Plateforme CityWatch avec la collaboration du Product
Owner et de fixer le nombre des itérations et leur durée. Rappelons que
l'itération 0 à durée trois jours, du 20 février 2017 au 22 février 2017.
