\begin{abstract}
    city watch est une idée de start-up qui offre une seule platforme qui vises une variété
de secteurs dans le but de donner des services de consulting.
Son démarche est diviser sur trois étapes :
Collection des informations, soit automatique tel que la detection des secousses,
embouteillages, signal ou vitesse via mobile sensors ou Interactive ( faire des rapports en
utilisant des formulaires) comme par exemples violence ou pollution.
Puis on analyse les informations collecté soit schématiquement ou statistiquement.
Pour créer un service de consulting basé sur le profiling, extraction du comportement
des utilisateurs ainsi leurs expériences.
\end{abstract}

\clearpage

\begin{otherlanguage}{english}
    \begin{abstract}
The aim of the project is to design and implement a unique platform named
CityWatch for user profiling using collected data from user behaviours in the
road.

The
The aim of the project is to investigate the performance of Gismos and
to design and construct a super multi-functional Gismo.

The novel aspects of the new Gismo are described. The abstract should
perhaps be about half a page long.

The results of testing, which show the abject failure of the Gismo,
are presented.

In the conclusions proposals for rectifying the deficiences are
outlined.
    \end{abstract}
\end{otherlanguage}
