\section{Itération 2: ( 3/8/2017 - 3/28/2017 )}

\subsection{L'objectif du sprint}
La méthodologie Scrum a eu un impact positif sur les développeurs, du point de vue social, elle
a valorisé le travail en équipe, la solidarité, le respect et la communication entre toutes les
parties prenantes (client, développeurs...). Elle a aussi changé leur vison sur le développement
des logiciels.\\
L'objectif de cette itération est d'ajouter un système de gestion des rapport et d'améliorer
la qualité de gestion de la page ``Dashboard''.
  \subsubsection{Planification de l'itération 2}
  Au cous de la réunion de planification nous avons sélectionné les tâches à réaliser au
  cours de cette itération tout en accord avec le ”Product-Owner”.
  \subsubsection{Préparation de la liste des tâches}
 \begin{center}
    \footnotesize
    \begin{longtable}{| p{1cm} | p{5cm} | p{7cm} | p{1cm} |}
        \caption{Taches à faire de la première itération}
        \label{tab:sprint2-backlog} \\

 \hline
 \multicolumn{1}{|c}{\textbf{Réf}} &
 \multicolumn{1}{|c}{\textbf{Spécification}} &
 \multicolumn{1}{|c}{\textbf{Description}} &
 \multicolumn{1}{|c|}{\textbf{Priorité}} \\ \hline
 \endhead

 \hline \multicolumn{4}{|r|}{{Continué en page suivante$\dotsc$}} \\ \hline
 \endfoot

 \hline \hline
 \endlastfoot

\hline
1 & Recherche trajectoire & Notion de trajectoirec & 1 \\ \hline
2 & Affichage du trajet sur la carte&Trajectoire affichée sur la carte d'un ID  & 1 \\ \hline
3 &Réception des données des ralentisseur &Table qui contient les informations des ralentisseur & 1 \\ \hline
4&Responsive design& IHM adaptable & 2 \\ \hline
5 & Ajout un bouton ralentisseur &Enable,Disabled respecte l'IHM  & 2 \\ \hline
6 & Filtrage des marqueurs dans la carte & légende simplifier & 1 \\ \hline
7 & chargement de l'image & Image temporaire dans le serveur jusqu'à la validation & 1 \\ \hline
8 & Enregistrement des informations du rapport dans la BD rapport& Enregistrer les données lors de la validation ou suppression après time-out & 1 \\ \hline
9 & Recherche test unitaires & Comment simplifier le travail en utilisant le test unitaire cote serveur  & 1 \\ \hline
10 & Recherche framework PHP & & 2 \\ \hline 
11 &Groupement des secousse sur la carte & Regrouper les secousses lors d'un zoom out sur la carte & 3 \\ \hline
\end{longtable}
\end{center}

  \subsubsection{Estimation de la deuxième itération}
  Comme l'iteration précédant, Nous avons fixé la période de cette itération à 3 semaine
  \begin{table}[htbp]
    \centering
    \begin{tabular}{| c | c | c | c |}
\hline
\textbf{Membre} & \textbf{Nombre d'heures par jour} & \textbf{Nombre de jours présent} & \textbf{Total en heures} \\ \hline
\hline

Moez & 8 & 18& 144\\ \hline
Rihab & 8 & 18 & 144 \\ \hline
\multicolumn{2}{c|}{} & \textbf{Total} & 288 \\ \cline{3-4}
    \end{tabular}
    \caption{Nombre d'heures de travail estimé de l'itération 2}
    \label{tab:sprint2-capacity}
\end{table}
\begin{center}
    \begin{longtable}{| l | l | l |}
        \caption{Nombre d'heures estimé pour la réalisation des taches}
        \label{tab:sprint2-estimation} \\

 \hline
 \multicolumn{1}{|c}{\textbf{Spécification}} &
 \multicolumn{1}{|c}{\textbf{Membre}} &
 \multicolumn{1}{|c|}{\textbf{Heures}} \\ \hline
 \endhead

 \hline \multicolumn{3}{|r|}{{Continué en page suivante$\dotsc$}} \\ \hline
 \endfoot

 \hline \hline
 \endlastfoot

\hline
Recherche trajectoire & Rihab & 5 x 2 \\ \hline
Affichage du trajet sur la carte& Moez & 13 x 2 \\ \hline
Réception des données des ralentisseur& Moez & 5 \\ \hline
Responsive design & Rihab & 5 x 2 \\ \hline
Ajout un bouton ralentisseur& Rihab & 13 x 2 \\ \hline
 Filtrage des marqueurs dans la carte  & Rihab & 13 \\ \hline
chargement de l'image & Moez & 5 \\ \hline
Enregistrement des informations du rapport dans la BD rapport & Moez & 5 \\ \hline
 Recherche test unitaires & Moez & 5 \\ \hline
Recherche framework PHP & Moez & 5 \\ \hline
Groupement des secousse sur la carte  & Rihab & 5 \\ \hline
\end{longtable}
\end{center}
  
\subsection{Backlog du sprint}
\subsection{Mises des normes}
\subsection{Modélisation UML}
\subsection{Évaluation suivant les normes mise}
\subsubsection{Produit de l'itération}
A la fin de l'itération 2,nous détaillons les différentes spécifications qui caractérisent et 
implémenté le systeme de gestion des rapport.
\paragraph{Page ``Rapport''}
\TODO{PAGE RAPPORT}
\paragraph{Page ``Dashboard''}
L'utilitaire de regroupement de marqueurs vous aide à gérer plusieurs marqueurs à différents niveaux de zoom.
Précisément, les  marqueurs  sont en fait des éléments à ce stade et ne deviennent réellement des
marqueurs qu'après leur rendu. Par souci de clarté, nous ne parlerons que de marqueurs 
dans ce document.

Lorsqu'un utilisateur affiche la carte à un niveau de zoom élevé comme montre la figure \ref{fig:sprint2-dashboard-screenshot1}, 
les différents marqueurs s'affichent sur 
la carte. Lorsqu'il effectue un zoom arrière comme montre la figure \ref{fig:sprint2-dashboard-screenshot},
les marqueurs se regroupent pour faciliter la consultation de la carte.
\begin{figure}[htbp]
  \centering
  \includegraphics[width=0.6\textwidth]{sprint2-dashboard-screenshot1}
  \caption{Groupement des marqueurs des secousses activé en un niveau de zoom bas}
  \label{fig:sprint2-dashboard-screenshot1}
\end{figure}
\begin{figure}[htbp]
  \centering
  \includegraphics[width=0.6\textwidth]{sprint2-dashboard-screenshot2}
  \caption{Groupement des marqueurs des secousses désactivé en un niveau de zoom haut  }
  \label{fig:sprint2-dashboard-screenshot2}
\end{figure}


\paragraph{"Application Mobile ``CityWatch''}
\TODO{App}
\subsubsection{Avis du ProductOwner}
\subsection{Travail contribué}
