\chapter{Annexe}
\pagebreak

\section{Documentation des Services Web d'itération 1}

Dans les pages suivantes, on décommente l'API RESTful résultante du notre
travail pendant l'itération 1. Quelques exemples du réquetes sont ajoutés à la
fin de chaque documentation du service web.

\clearpage
\subsection{Service Post Position}
\label{appendix:sprint1-position-post-doc}

\textbf{POST} \ \texttt{/positions/\textit{:id}}

\begin{table}[htbp]
    \centering
    \begin{tabularx}{\textwidth}{llll}
        \multicolumn{4}{X}{\textbf{Paramètres Service Post Position}} \\
        \toprule
        \rowcolor{gray!20}
        \multicolumn{1}{c}{\textbf{Élément}} &
        \multicolumn{1}{c}{\textbf{Obligation}} &
        \multicolumn{1}{c}{\textbf{Type}} &
        \multicolumn{1}{c}{\textbf{Description}} \\
        \midrule
        \verb|:id| & obligatoire & string & identificateur du périphérique \\
        \verb|lat| & obligatoire & nombre [-90, 90] & latitude de la position \\
        \verb|lng| & obligatoire & nombre [-180, 180] & longitude de la position \\
        \verb|last_modified| & optionnel & string (RFC3339) & date du capture de la position \\
        \bottomrule
    \end{tabularx}
\end{table}

\textbf{Réponse Codes}

\begin{description}
    \item[\texttt{204 No Content}] Position était crée ou mette à jour avec succès.
    \item[\texttt{400 Bad Request}] Le contenu JSON n'est pas valide.
    \item[\texttt{422 Unprocessable Entity}] Entité obligatoire absente ou format d'une entité est invalide.
\end{description}

\textbf{Entités de réponse:} n/a

\begin{listing}[H]
    %\caption*{Démonstration Service Post Position}
    \begin{minted}[label={Exmple Requéte}]{bash}
curl -i -X POST http://localhost/positions/1 \
     -d '{"lat": 10.71979600000000,
          "lat": "34.72563400000000",
          "last_modified": "2017-03-27 10:34:01"}' \
     -H 'Content-Type: application/json'
\end{minted}
\begin{minted}[label={Réponse en Succés}]{http}
HTTP/1.1 200 Success
Content-Type: application/json

{
    "success": true
}
\end{minted}
\end{listing}

\clearpage
\subsection{Service Get Position}
\label{appendix:sprint1-position-get-doc}

\textbf{GET} \ \texttt{/positions/\textit{:id}}

\begin{table}[htbp]
    \centering
    \begin{tabularx}{\textwidth}{llll}
        \multicolumn{4}{X}{\textbf{Paramètres Service Get Position}} \\
        \toprule
        \rowcolor{gray!20}
        \multicolumn{1}{l}{\textbf{Élément}} &
        \multicolumn{1}{l}{\textbf{Obligation}} &
        \multicolumn{1}{l}{\textbf{Type}} &
        \multicolumn{1}{l}{\textbf{Description}} \\
        \midrule
        \verb|:id| & obligatoire & string & identificateur du périphérique \\
        \bottomrule
    \end{tabularx}
\end{table}

\begin{table}[htbp]
    \centering
    \begin{tabularx}{\textwidth}{lll}
        \multicolumn{3}{X}{\textbf{Entités du réponse du Service Post Position}} \\
        \toprule
        \rowcolor{gray!20}
        \multicolumn{1}{l}{\textbf{Élément}} &
        \multicolumn{1}{l}{\textbf{Type}} &
        \multicolumn{1}{l}{\textbf{Description}} \\
        \midrule
        \verb|lat| & nombre [-90, 90] & latitude de la position \\
        \verb|lng| & nombre [-180, 180] & longitude de la position \\
        \bottomrule
    \end{tabularx}
\end{table}

\begin{table}[htbp]
    \centering
    \begin{tabularx}{\textwidth}{lll}
        \multicolumn{3}{X}{\textbf{En-tête Réponse Service Get Position}} \\
        \toprule
        \rowcolor{gray!20}
        \multicolumn{1}{l}{\textbf{En-tête}} &
        \multicolumn{1}{l}{\textbf{Type}} &
        \multicolumn{1}{l}{\textbf{Description}} \\
        \midrule
        \verb|Last-Modified| & string (RFC1123) & Date du dernier modification du position \\
        \bottomrule
    \end{tabularx}
\end{table}

\textbf{Réponse Codes}

\begin{description}
    \item[\texttt{200 Success}] Dernière position du périphérique spécifié retournée.
    \item[\texttt{404 Not Found}] Aucune position trouvée pour le périphérique spécifié.
\end{description}

\begin{listing}[H]
    %\caption*{Démonstration Service Get Position}
    \begin{minted}[label={Exemple Requéte}]{bash}
curl -i http://localhost/api/v1/positions/1
\end{minted}
\begin{minted}[label={Réponse en Succés}]{http}
HTTP/1.1 200 OK
Content-Type: application/json
Last-Modified: Wed, 17 May 2017 14:51:00 GMT

{
    "lng": "10.71989600000000",
    "lat": "34.72526400000000"
}
\end{minted}
\end{listing}

\clearpage
\section{Documentation des Services Web d'itération 2}

\TODO{doc}

\clearpage
\section{Documentation des Services Web d'itération 3}

\TODO{doc}

\clearpage
\chapter{Annexe}

\section{Evolution du qualité du code}

\TODO{use Hits by Response Code chart from server state to show code quality improvement}
