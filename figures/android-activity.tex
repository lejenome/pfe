% Diagram of Android activity life cycle
% Author: Pavel Seda 
% Drawing part, node distance is 1.5 cm and every node
% is prefilled with white background
\begin{figure}[H]
 \centering
 \footnotesize

\begin{tikzpicture}[node distance=1.5cm,
    every node/.style={fill=white, font=\sffamily}, align=center]
  % Specification of nodes (position, etc.)
  \node (start)             [activityStarts]              {L'activité démarre};
  \node (onCreateBlock)     [process, below of=start]          {onCreate()};
  \node (onStartBlock)      [process, below of=onCreateBlock]   {onStart()};
  \node (onResumeBlock)     [process, below of=onStartBlock]   {onResume()};
  \node (activityRuns)      [activityRuns, below of=onResumeBlock]
                                                      {Activity is running};
  \node (onPauseBlock)      [process, below of=activityRuns, yshift=-1cm]
                                                                {onPause()};
  \node (onStopBlock)       [process, below of=onPauseBlock, yshift=-1cm]
                                                                 {onStop()};
  \node (onDestroyBlock)    [process, below of=onStopBlock, yshift=-1cm] 
                                                              {onDestroy()};
  \node (onRestartBlock)    [process, right of=onStartBlock, xshift=4cm]
                                                              {onRestart()};
  \node (ActivityEnds)      [startstop, left of=activityRuns, xshift=-4cm]
                                                        {Le processus est tué};
  \node (ActivityDestroyed) [startstop, below of=onDestroyBlock]
                                                    {l'activité est arrêtée};     
  % Specification of lines between nodes specified above
  % with aditional nodes for description 
  \draw[->]             (start) -- (onCreateBlock);
  \draw[->]     (onCreateBlock) -- (onStartBlock);
  \draw[->]      (onStartBlock) -- (onResumeBlock);
  \draw[->]     (onResumeBlock) -- (activityRuns);
  \draw[->]      (activityRuns) -- node[text width=4cm]
                                   {Une autre activité s'intercole devent notre activité} (onPauseBlock);
  \draw[->]      (onPauseBlock) -- node {Notre activité n'est plus visible}
                                   (onStopBlock);
  \draw[->]       (onStopBlock) -- node {L'activité est arrêtée par le système ou l'utilisateur} (onDestroyBlock);
  \draw[->]    (onRestartBlock) -- (onStartBlock);
  \draw[->]       (onStopBlock) -| node[yshift=1.25cm, text width=3cm]
                                   {L'activité revientsur le devant de la scène}
                                   (onRestartBlock);
  \draw[->]    (onDestroyBlock) -- (ActivityDestroyed);
  \draw[->]      (onPauseBlock) -| node(priorityXMemory)
                                   {Priorité élevée $\rightarrow$ plus mémoire}
                                   (ActivityEnds);
  \draw           (onStopBlock) -| (priorityXMemory);
  \draw[->]     (ActivityEnds)  |- node [yshift=-2cm, text width=3.1cm]
                                    {L'utilisateur retourne vers l'activité}
                                    (onCreateBlock);
  \draw[->] (onPauseBlock.east) -- ++(2.6,0) -- ++(0,2) -- ++(0,2) --
     node[xshift=1.2cm,yshift=-1.5cm, text width=2.5cm]
     {L'activité revient sur le devant de la scéne}(onResumeBlock.east);

  \end{tikzpicture}
  \caption{Diagramme de cycle de vie d'activite Android}
  \captionsource{Pavel Seda, \TeX example.net [Modifié]}{http://www.texample.net/tikz/examples/android/}
  \label{fig:android-activity}
\end{figure}
