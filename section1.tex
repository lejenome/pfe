\section{Itération 1: Système de tracking}

\subsection{L'objectif du sprint}
Dans le but de définir le périmètre fonctionnel du Sprint 1 et faire
sa planification, tous les membres de l'équipe ont participé à un réunion
avec le Scrum Master et le Product Owner pour définir le backlog du sprint.\\

Une fois la première ébauche du Backlog est réalisée, le product owner peut découper
les spécifications de haut niveau vers des spécifications plus raffinées. Dès lors nous
passons à la planification d’une itération. Il est important de rappeler que les premières
spécifications du Backlog doivent être:
\begin{itemize}
 \item assez précise pour être estimée par l’équipe
 \item assez petite pour être développée et testée durant une itération
\end{itemize}
Ayant une bonne idée sur le produit ainsi que l’objectif à atteindre, l’équipe et le
product-owner peuvent passer à l’élaboration des itérations.
Dans notre projet
``Plateforme CityWatch''
la durée d’une Itération est fixée à trois
semaines (170h par itération). Dans ce chapitre nous décrivons le déroulement des deux
premières itérations. Une itération commence par sa planification et finit par être revue
avec ”Product-Owner”.
\subsubsection{Planification de l'itération 1}
La planification de l’itération représente une étape importante dans la vie de
l’itération parce que au cours du planification, on divise l’itération a plusieurs étapes
pour mieux atteindre le résultats attendu à la fin de l’itération.
Dans cette première itération on a décidé de diviser cette dernière en trois grandes
parties une partie consacrée pour la création de la base de donnée et la génération du
model, la deuxième partie est consacré pour la réalisation de la page d’accueil, et une
troisième partie qui répond aux besoins des autres tâches de cette itérations
\subsubsection{But de la première itération}
Le but de cette itération est d'étudier notre Serveur ,de générer notre modèle,de réaliser
la page Web (dashboard) et développer les vues nécessaires pour permettre à l'utilisateur 
de consulter la dernière position du véhicule selon les spécifications du Backlog. 
\subsection{Backlog du sprint}
Le sprint Backlog est un outil qui facilite la récupération des tâches et qui fait
la mise au point du travail tout en précisant les taches que contient chaque 
user-story du Product Backlog.
Le Backlog agrée pour ce sprint est:
%\begin{table}
%\begin{tabular}{| l | l | l | l | l | l |}
%\hline
 %ID & Cas d'utilisations & En tant qu' & Je veux qu' & Pour & Priorité \\ \hline
 %1 & Authentification mobile & Utilisateur & XXXXXX & XXXXXX & XXXXXX \\ \hline
 %\end{tabular}
%\caption{tableau 1}
%\end{table}
\subsection{Mises des normes}


\subsection{Modélisation UML}
\subsection{Évaluation suivant les normes mise}

Dans 1\ier{} phase, on a implementé la method POST du resource Postion,

\subsection{Travail contribué}

