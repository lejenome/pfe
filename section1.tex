\section{Itération 1: Système de tracking}

\subsection{L'objectif du sprint}
Dans le but de définir le périmètre fonctionnel du Sprint 1 et faire
sa planification, tous les membres de l'équipe ont participé à un réunion
avec le Scrum Master et le Product Owner pour définir le backlog du sprint.\\

Une fois la première ébauche du Backlog est réalisée, le product owner peut découper
les spécifications de haut niveau vers des spécifications plus raffinées. Dès lors nous
passons à la planification d’une itération. Il est important de rappeler que les premières
spécifications du Backlog doivent être:
\begin{itemize}
 \item assez précise pour être estimée par l’équipe
 \item assez petite pour être développée et testée durant une itération
\end{itemize}
Ayant une bonne idée sur le produit ainsi que l’objectif à atteindre, l’équipe et le
product-owner peuvent passer à l’élaboration des itérations.
Dans notre projet ``Plateforme CityWatch'' la durée d’une Itération est fixée à trois
semaines (170h par itération). Dans ce chapitre nous décrivons le déroulement des deux
premières itérations. Une itération commence par sa planification et finit par être revue
avec ”Product-Owner”.
\subsubsection{Planification de l'itération 1}
La planification de l'itération représente une étape importante dans la vie de
l'itération parce que au cours du planification, on divise l'itération a plusieurs étapes
pour mieux atteindre le résultats attendu à la fin de l'itération.
Dans cette première itération on a décidé de diviser cette dernière en trois grandes
parties une partie consacrée pour la création de la base de donnée et la génération du
modèle, la deuxième partie est consacré pour la réalisation de la page d’accueil, et une
troisième partie qui répond aux besoins des autres tâches de cette itérations
\subsubsection{But de la première itération}
Le but de cette itération est d'étudier notre Serveur ,de générer notre modèle,de réaliser
la page Web (dashboard) et développer les vues nécessaires pour permettre à l'utilisateur
de consulter la dernière position du véhicule selon les spécifications du Backlog.
\subsection{Backlog du sprint}
Le sprint Backlog est un outil qui facilite la récupération des tâches et qui fait
la mise au point du travail tout en précisant les taches que contient chaque
user-story du Product Backlog.
Le Backlog agrée pour ce sprint est:

\TODO{sprint backlog}

\subsection{Mises des normes}

L'exigence de l'api RESTful est:
\begin{itemize}
    \item Performance: Le temps de réponse doit être en une durée raisonnable.
    \item Multi-Utilisateurs: Les positions des différents utilisateurs sont
        distinguables.
    \item Fiabilité: L'implémentation doit vérifier la structure des données
        reçus, la disponibilité des champs obligatoires et leurs formats.
    \item Portabilité: Le système doit support la différence en zone du temps
        et en internalisation, format de présentation des données géologique et
        sa précision.
\end{itemize}

Ces caractéristique ont étés adressés pendant l'implémentation du service web.

\subsubsection{Performance}

On a minimisé la dépendance en bibliothèques externes et extensions PHP.
De plus, Le nombre des requêtes SQL ont étés minimisés à une seule requêtes pour
chaque méthode du notre service web.

\subsubsection{Multi-Utilisateurs}

Les positions à enregistrer doivent contenus un id qui doit etre utiliser aussi
pour retirer la dernière position.
On a utilisé diffèrent identificateur pour chaque périphériques basé sur le MAC
du phone.

\subsubsection{Portabilité}

On a utilisé la format standardisée JSON pour le transfert de données ce qui
assure que la portabilité des représentations des différents types des données
(nombres, booléens, strings, $\dotsc$) indépendant du localisation.

Pour le représentation des dates dans le content du requêtes HTTP, on a utilisé
la format RFC3339 \cite{RFC3339} avec UTC comme la défaut zone du temps.
Pour les en-têtes du requêtes HTTP, la format de représentation des dates est
RFC1123 \cite{RFC1123}.

La format du présentation des positions géologique (latitude et longitude)
choisi est la même représentation des nombres en JSON avec une précision jusqu'à
14 chiffres après le virgule. Les valeurs envoyés doivent respecte l'intervalle
des diffèrent entités géologiques ($latitude \in [-90, 90]$,
$longitude \in [-180, 180]$)

\subsubsection{Fiabilité}

Pour assuré la fiabilité d'un service web, on doit vérifier la validité de
contenu reçu même si envoyé depuis un source de confiance avant de les
utiliser. La vérification inclue le test de disponibilité des entités
obligatoires et le test de leurs formats.

\subsection{Modélisation UML}
\subsection{Évaluation suivant les normes mise}
\subsubsection{Page Web ``Dashboard''}
Dans cette vue la page principale de Plateforme s'affiche. Cette page doit être
bien compréhensible pour l'utilisateur.

\TODO{sprint1 dashboard screenshot}

\subsubsection{Implementé la method POST}

Dans 1\ier{} phase, on a implémenté la méthode POST du ressource Position,

\TODO{RESTful API design describtion as http-api-design-en.pdf}

\subsection{Travail contribué}

