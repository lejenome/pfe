\section{Itération 1: Système de tracking}

\subsection{L'objectif du sprint}
Dans le but de définir le périmètre fonctionnel du Sprint 1 et faire
sa planification, tous les membres de l'équipe ont participé à un réunion
avec le Scrum Master et le Product Owner pour définir le backlog du sprint 
\subsection{Backlog du sprint}
Le sprint Backlog est un outil qui facilite la récupération des tâches et qui fait
la mise au point du travail tout en précisant les taches que contient chaque 
user-story du Product Backlog.
\footnote{pepep}
Le Backlog agrée pour ce sprint est:
%\begin{table}
%\begin{tabular}{| l | l | l | l | l | l |}
%\hline
 %ID & Cas d'utilisations & En tant qu' & Je veux qu' & Pour & Priorité \\ \hline
 %1 & Authentification mobile & Utilisateur & XXXXXX & XXXXXX & XXXXXX \\ \hline
 %\end{tabular}
%\caption{tableau 1}
%\end{table}
\subsection{Mises des normes}
\subsection{Modélisation UML}
\subsection{Évaluation suivant les normes mise}
\subsection{Travail contribué}

