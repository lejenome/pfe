\chapter*{Introduction}
\addcontentsline{toc}{chapter}{Introduction}

La méthode de gestion de projet adoptée dans le projet présent est la
méthodologie Scrum qui se base sur des itérations (Sprint) de même duré de
temps et qui constitue l'avancement en étapes du projet. Dans chaque itération,
le cycle de vue d'un mini projet va être traité. C'est-à-dire dés la
planification du mini projet, l'estimation des taches, la conception, le
développement, le test, l'intégration jusqu'à la documentation.

Avant de commencer à élaborer notre projet, une période du temps était
consacrée pour préparer tout qui était nécessaire pour le lancement dans des
bonne condition.  Cette période est souvent nommée ``Itération 0''. Elle est
dédiée généralement à la recherche des choix techniques, à la mise en place de
l'environnement de développement et aussi la définition de carnet du produit
qui présente la somme des fonctionnalités du système à réaliser qui est nommé
``Backlog Général''.
